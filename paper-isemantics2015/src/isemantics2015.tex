%%%%%%%%%%%%%%%%%%%%%%%%%%%%%%%%%%%%%%%%%%%%%%%%%%%%%%%%%%%%%%%%%%%%%%%%%%%%%%%%%%%%%%%%%%%%%%%%%%%%%%%%%%
%%%   %%%%
%%%%%%%%%%%%%%%%%%%%%%%%%%%%%%%%%%%%%%%%%%%%%%%%%%%%%%%%%%%%%%%%%%%%%%%%%%%%%%%%%%%%%%%%%%%%%%%%%%%%%%%%%%%

\documentclass{sig-alternate}

\usepackage[utf8]{inputenc}
\usepackage{amssymb}
\setcounter{tocdepth}{3}
\usepackage{graphicx}
\usepackage{tabularx}
\usepackage{url}
\usepackage{listings}
\usepackage{subfigure}
\usepackage{algorithmic}
\usepackage{algorithm}


%\newcommand{\keywords}[1]{\par\addvspace\baselineskip
%\noindent\keywordname\enspace\ignorespaces#1}

% todo macro
\usepackage{color}
\newtheorem{deflda}{Axiom}
\newcommand{\todo}[1]{\noindent\textcolor{red}{{\bf \{TODO}: #1{\bf \}}}}
\newcommand{\neon}{NeOn }
\newcommand{\protege}{Prot{\'e}g{\'e} }



%%%%%%%%%%%%%%%%%%%%%%%%%%%%%%%
%%%  Beginning of document  %%%
%%%%%%%%%%%%%%%%%%%%%%%%%%%%%%%

\begin{document}

% first the title is needed
% --- ACM copyright metadata here ---
%\conferenceinfo{SEM}{14, September 04 - 05 2014, Leipzig, AA, Germany}
%\CopyrightYear{2015}
%\crdata{978-1-4503-2927-9/14/09.}

\title{How to access and reuse ontologies in real-world scenarios}

\numberofauthors{3}
\author{
% 1st. author
\alignauthor Ghislain Auguste Atemezing\\
       \affaddr{MONDECA}\\
       \affaddr{35 boulevard Strasbourg, Paris, France}\\
       \email{\small{ghislain.atemezing@mondeca.com}}
% 3rd author
  \alignauthor Nuria Garc{\'i}a-Santa\\
       \affaddr{Expert System Iberia}\\
       \affaddr{Campo de las Naciones, Madrid, Spain.}\\
       \email{\small{ngarcia@expertsystem.com}}
% 2nd. author
\alignauthor Boris Villaz{\'o}n-Terrazas\\
       \affaddr{Expert System Iberia}\\
       \affaddr{Campo de las Naciones, Madrid, Spain.}\\
       \email{\small{bvillazon@expertsystem.com}}       
}

\maketitle


%%%%%%%%%%%%%%%%%%
%%%  Abstract  %%%
%%%%%%%%%%%%%%%%%%

\begin{abstract}
Developing ontologies, by reusing already available and well-known ontologies, is commonly acknowledge to play a crucial role to facilitate inclusion and expansion of the Web of Data. Some recommendations exist to guide ontologists in ontology engineering, but they do not provide guidelines on how to reuse vocabularies at low fine grained, i.e., reusing specific classes and properties. Moreover, it is still hard to find a tool that provides users with an environment to reuse terms. This paper presents Prot{\'e}g{\'e}LOV, a plugin for the ontology editor Prot{\'e}g{\'e}, that combines the access to the Linked Open Vocabularies (LOV) during ontology modeling. It allows users to search a term in LOV and provides three actions if the term exists : (i) replace the selected term in the current ontology; (ii) add the {\tt rdfs:subClassOf} or {\tt rdfs:subPropertyOf} axiom between the selected term and the local term; and (iii) add the {\tt owl:equivalentClass} or {\tt owl:equivalentProperty} between the selected term and local term. Results from a preliminary user study indicate that Prot{\'e}g{\'e}LOV does provide an intuitive access and reuse of terms in external vocabularies.
%\keywords{ontology engineering, ontology reuse, LOV, Prot{\'e}g{\'e}, Plugin} 
\end{abstract}

% Categories for the papers.
\todo{Find the right one here}
\category{H.4}{Information Systems Applications}{Miscellaneous}
\category{H.3.5}{Online Information Services}{Data sharing}[Web-based services]

\terms{Accessibility, Linked Data, Vocabularies}

\keywords{ontology engineering, ontology reuse, LOV, Prot{\'e}g{\'e}, Plugin}



%%%%%%%%%%%%%%%%%%%%%%%%%
%%%  1. Introduction  %%%
%%%%%%%%%%%%%%%%%%%%%%%%%

%\vspace{-3mm}
\section{Introduction}\label{sec:introduction}
%\todo{we have to extend this Related work}

So far, Linked Data principles and practices are being adopted by an increasing number of data providers, getting as result a global data space on the Web containing hundreds of LOD datasets \cite{Heath_Bizer_2011}. There are already several guidelines for generating, publishing, interlinking, and consuming Linked Data \cite{Heath_Bizer_2011}. An important task, within the generation process, is to build the vocabulary to be used for modelling the domain of the data sources, and the common recommendation is to reuse as much as possible available vocabularies \cite{Heath_Bizer_2011,hyland14}. This reuse approach speeds up the vocabulary development, and therefore, publishers will save time, efforts, and resources. 

There are research efforts, like the NeOn Methodology \cite{suarezfigueroa2012ontology}, the Best Practices for Publishing Linked Data - W3C Working Group Note \cite{hyland14}, and the work proposed by Lonsdale et al. \cite{Lonsdale2010318}. However, at the time of writing we have not found specific and detailed guidelines that describe how to reuse available vocabularies at fine granularity level,i.e., reusing specific classes and properties. Our claim is that this difficulty in how to reuse vocabularies at fine grained level is one of major barriers to the reuse of vocabularies on the Web and in consequence to deployment of Linked Data.

Moreover, the recent success of Linked Open Vocabularies (LOV\footnote{\url{http://lov.okfn.org/dataset/lov/}}) as a central point for curated catalog of ontologies is helping to convey on best practices to publish vocabularies on the Web, as well as to help in the Data publication activity on the Web. LOV comes with many features, such as an API, a search function and a SPARQL endpoint.

In this paper we propose a set of guidelines for this task, and provide technological support by means of a plugin for \protege, which is one of the popular frameworks for developing ontologies in a variety of formats including OWL, and RDF(S). It is backed by a strong community of developers and users in many domains. One success on \protege also depends on the availability to extend the core framework adding new functionalities by means of plug-ins. In addition, we propose to explore, design and implement a plug-in of LOV in \protege for easing the development of ontologies by reusing existing vocabularies at fine grained level. The tool helps to improve the modeling and reuse of ontologies used in the LOD cloud.

%, by providing the following features in Prot{\'e}g{\'e}:
%\begin{itemize}
%\item Import easily vocabularies from LOV into \protege. 
%\item Propose to the user a list of candidate vocabularies in LOV matching the term
%\item Have an updating mechanism (synchronization) to LOV catalog
%\item Check if a new vocabulary created in \protege satisfied the LOV recommendations \cite{pybernard12}
%\item Suggest to LOV a new created vocabulary within \protege.
%\end{itemize}

%This paper presents \protege, a first implementation of the LOV realized as a plugin for the ontology editor \protege. 





  






%%%%%%%%%%%%%%%%%%%%%%%%%%%%%%%%%%%%%%%%%%%%%%%%%%
%%%  2. Related work  %%%
%%%%%%%%%%%%%%%%%%%%%%%%%%%%%%%%%%%%%%%%%%%%%%%%%%
%\vspace{-3mm}
\section{Related Work}\label{sec:soa}
%\vspace{-3mm}
%Show here some work related to plugin for helping in ontology development:Neon and other
%\todo{target to guidelines}
\todo{I suggest to move this related work at the end}
In the literature there exist many attempts to advise vocabulary publishers on the importance of reusing terms, as indicated in \cite{janowicz2014five,jimenez2008}. However, to the best of our knowledge there are not guidelines to help vocabulary practitioners to reuse vocabularies in real-world scenario, and considering specific ontology/vocabulary elements. 

In the W3C Government Linked Data best practice document \cite{hyland14}, reusing vocabularies is recommended by providing to stakeholders a basic checklist when using or extending a vocabulary. It gives general guidance to follow before publishing the vocabulary, not guidance during the creation of the vocabulary. Our proposal is to guide the users during the implementation process.

Recently, Janowicz et al. have propopsed a 5 stars rating for Linked (Open) Data vocabulary use to ``encourgage data owners, engineers and practitioners to publish and use vocabularies on the Web'' \cite{janowicz2014five}. They make it clear that the rating do not refer to the qualitiy of the vocabularies. In the definition of the rating system, the third star is given to a vocabulary linked to other vocabularies by means of explicit alignments and import of external vocabularies. Our guidelines make it easier to vocabulary publishers to obtain at least 3-star vocabularies.

Other initiatives similar to the tool we have developed can be found in the literature but not currently maintained. The BioPortal Reference Plugin\footnote{\url{http://protegewiki.stanford.edu/wiki/BioPortal_Reference_Plugin}} allows the user to insert into the ontology references to external ontologies and terminologies stored in BioPortal\footnote{\url{http://bioportal.bioontology.org/}}. The plugin allows to generate external reference of a selected term. Additionally, the BioPortal Import Plugin\footnote{\url{http://protegewiki.stanford.edu/wiki/BioPortal_Import_Plugin}} allows users to import classes from external ontologies stored in the BioPortal ontology repository. The user can import entire trees of classes with a desired depth and choose which properties to import for each class. However, those plugins work only with \protege 3.x releases and are not ported yet to recent versions. 

Most closely related to the {\protege}LOV plugin are approaches that use semantic search engine to support the process of editing an ontology and make large scale knowledge reuse automatically integrated in the tool. An example is the Watson Plugin \cite{neonguide2008} for the \neon Toolkit \cite{haase2008neon}, a plugin supporting the \neon life-cycle management using the Watson \cite{d2007watson} APIs\footnote{\url{http://watson.kmi.open.ac.uk/WS_and_API.html}}.However the similar plugin for Protege\footnote{\url{http://protegewiki.stanford.edu/wiki/Watson_Search_Preview}} is just a proof of concept rather than a real plugin.

 



%%%%%%%%%%%%%%%%%%%%%%%%%%%%%%%%%%%%%%%%%%%%%%%%%%%
%%%  3.LOV APIs Description  %%%
%%%%%%%%%%%%%%%%%%%%%%%%%%%%%%%%%%%%%%%%%%%%%%%%%%%
%\vspace{-3mm}
%\section{Linked Open Vocabulaires (LOV)}\label{sec:lov}
\section{Technological Support}
%\vspace{-3mm}
Talk about the LOV catalog and APIs..





%%%%%%%%%%%%%%%%%%%%%%%%%%%%%%%%%%%%%%%%%%%%%%%%%%%
%%%  4.ProtegeLOV Description  %%%
%%%%%%%%%%%%%%%%%%%%%%%%%%%%%%%%%%%%%%%%%%%%%%%%%%%
%\vspace{-3mm}
%\section{Prot{\'e}g{\'e}LOV}\label{sec:classification}
%Explain our implementations.. views, controllers\\
Add a screenshot

\begin{figure}
\center
\includegraphics[scale=0.7]{img/LOVmockup.png}
\end{figure}




%%%%%%%%%%%%%%%%%%%%%%%%%%%%%%%%%%%%%%%
%%%  5. Conclusion and Future Work  %%%
%%%%%%%%%%%%%%%%%%%%%%%%%%%%%%%%%%%%%%%
%\vspace{-3mm}
\section{Evaluation}\label{sec:conclusion}
Small evaluation with few users? and compare with NeOn plugin maybe.



%%%%%%%%%%%%%%%%%%%%%%%%%
%%%  Acknowledgments  %%%
%%%%%%%%%%%%%%%%%%%%%%%%%
%\vspace{-1mm}
\paragraph{\textbf{Acknowledgments.}} %\label{sec:acknowledgments}
Thanks to Pierre-Yves and the LOV team for maintaining the LOV catalog and the API access.
% More acknowledgments here
%\vspace{-3mm}

\bibliographystyle{abbrv}
%\nocite{*}
\bibliography{isemanticsbib}
\balancecolumns
\end{document}
