%Show here some work related to plugin for helping in ontology development:Neon and other
\todo{target to guidelines}
In the literature there exist many attempts to advise vocabulary publishers on the importance of reusing terms, as indicated in \cite{janowicz2014five,jimenez2008}\todo{add more citation papers}. However, to the best of our knowledge there are not guidelines to help vocabulary practitioners to reuse vocabularies in real-world scenario, and considering specific ontology/vocabulary elements. 

The BioPortal Reference Plugin\footnote{\url{http://protegewiki.stanford.edu/wiki/BioPortal_Reference_Plugin}} allows the user to insert into the ontology references to external ontologies and terminologies stored in BioPortal\footnote{\url{http://bioportal.bioontology.org/}}. The plugin allows to generate external reference of a selected term. Additionally, the BioPortal Import Plugin\footnote{\url{http://protegewiki.stanford.edu/wiki/BioPortal_Import_Plugin}} allows users to import classes from external ontologies stored in the BioPortal ontology repository. The user can import entire trees of classes with a desired depth and choose which properties to import for each class. However, those plugins work only with \protege 3.x releases and are not ported yet to recent versions. 

Most closely related to the {\protege}LOV plugin are approaches that use semantic search engine to support the process of editing an ontology and make large scale knowledge reuse automatically integrated in the tool. An example is the Watson Plugin \cite{neonguide2008} for the \neon Toolkit \cite{haase2008neon}, a plugin supporting the \neon life-cycle management using the Watson \cite{d2007watson} APIs\footnote{\url{http://watson.kmi.open.ac.uk/WS_and_API.html}}.However the similar plugin for Protege\footnote{\url{http://protegewiki.stanford.edu/wiki/Watson_Search_Preview}} is just a proof of concept rather than a real plugin.

 
